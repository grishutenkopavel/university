% "Лабораторная работа 2"

\documentclass[a4paper,12pt]{article} % тип документа

% report, book

%  Русский язык

\usepackage[T2A]{fontenc}			% кодировка
\usepackage[utf8]{inputenc}			% кодировка исходного текста
\usepackage[english,russian]{babel}	% локализация и переносы


% Математика
\usepackage{amsmath,amsfonts,amssymb,amsthm,mathtools} 


\usepackage{wasysym}

\usepackage{hyperref}

%Заговолок
\author{кафедра КТиЭО}
\title{Работа с текстом в \LaTeX{}}
\date{\today}


\begin{document} % начало документа
\begin{enumerate}
    \item \url{https://www.ibm.com/developerworks/ru/library/latex_tutorial_01/}
    
    Библиотека компании IBM содержит различные обучающие материалы в том числе и по работе в LaTeX. Подойдет для тех людей которые ищут уроки по этой издательской системе.
    
    \item \url{http://dkhramov.dp.ua/Comp.CreatingAnArticleInLatex}
    
    Ресурс содержит инструкцию о том как оформить статью с помощью издательской системы LaTeX.
    
    \item \url{https://inf.1sept.ru/view_article.php?id=200801205}
    
    Статья содержит информацию о том, как использовать LaTeX для формирования математических формул. 
    
    \item \url{https://ru.coursera.org/lecture/latex/kak-rabotaiet-latex-lAJai}
    
    Бесплатный курс по издательской системе, который поможет легко обучится основным тонкостям работы с LaTeX.
    
    \item \url{https://ru.stackoverflow.com/questions/tagged/latex}
    
    Форум, который поможет найти информацию на любой вопрс, связанный с LaTex.
    
\end{enumerate}
\end{document} % конец документа