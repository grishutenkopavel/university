% "Лабораторная работа 2"

\documentclass[a4paper,12pt]{article} % тип документа

% report, book

%  Русский язык

\usepackage[T2A]{fontenc}			% кодировка
\usepackage[utf8]{inputenc}			% кодировка исходного текста
\usepackage[english,russian]{babel}	% локализация и переносы


% Математика
\usepackage{amsmath,amsfonts,amssymb,amsthm,mathtools} 


\usepackage{wasysym}

\usepackage{hyperref}

%Заговолок
\author{кафедра КТиЭО}
\title{Работа с текстом в \LaTeX{}}
\date{\today}


\begin{document} % начало документа
\begin{enumerate}
    \item Что означает WYSIWYG подход? 
    
    Дословно переводится "что ты видишь, то ты получаешь". Это подход набора информации, при котором результат виден сразу после его набора, при этом нет никаких лишних команд для компьютера, мешающих восприятию информации.
    
    \item Как создать файл для LaTeX? 
    
    Файл можно создать в любом текстовом редакторе. С помощью символов ascii формируется основное содержание документа включающих информацию и команды.
    
    \item Как писать спецсимволы и обратную косую черту?
     
    Для набора спецсимволов нужно использовать $\backslash$ перед ними. Чтобы написать обратную косую черту нужно вставить накую команду \$$\backslash$backslash\$.
    
    \item Как оставить коментарии для редактора?
    
    Для того чтобы написанный вами комментарий не отображался в документе нужно использовать символ \% перед вашим комментарием.
    
    \item Как задать тип создаваемого документа?
    
    $\backslash$documentclass[опции]\{класс\} - служит для задания типа документа, эту команду можно применить, например, следующим образом: $\backslash$documentclass[a4paper,12pt]\{article\}.
    
    \item Для чего нужна команда  $\backslash$usepackage?
    
     С помощью команды $\backslash$usepackage можно добавлять дополнительные пакеты расширений, например, поддержка математических формул.
    
    \item Что означает формат .dvi?
    
    Device Independent file ( файл не зависящий от устройства) -- это основной результат запуска LaTeX. Его содержимое можно увидеть в программе отображения DVI.
    
    \item Чем LaTeX отличается от других систем верстки?
    
    LaTeX отличается от других систем верстки в том, что вам нужно лишь задавать ему логическую и смысловую структуру текста. Затем он выбирает типографическую форму текста в соответствии с заданными командами.
    
    \item Как в LaTeX писать на русском языке?
    
    Для того чтобы выбрать нужный язык используется следующая команда: $\backslash$usepackage[язык]\{babel\}. Для русского языка она имеет вид:
     $\backslash$usepackage[russian]\{babel\}.
     
    \item Какой командой осуществляется ввод многоточия?
    
    Чтобы напечатать многоточие испльзуется комада $\backslash$ldots.
    
\end{enumerate}
\end{document} % конец документа