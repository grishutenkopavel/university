% "Инвариантное задание 5"

\documentclass[a4paper,12pt]{article} % тип документа

% report, book

%  Русский язык

\usepackage[T2A]{fontenc}			% кодировка
\usepackage[utf8]{inputenc}			% кодировка исходного текста
\usepackage[english,russian]{babel}	% локализация и переносы


% Математика
\usepackage{amsmath,amsfonts,amssymb,amsthm,mathtools} 


\usepackage{wasysym}

%Заговолок
\author{Гришутенко Павел, 1 группа}
\title{Команды создания текстовых документов в \LaTeX{}}
\date{\today}


\begin{document} % начало документа

\section{Команды для создания LaTeX документа}
\begin{enumerate}
    \item documentclass\{...\} -- команда, которая задает тип документа, она позволят, например, изменить формат листа документа.
    \item usepackage\{...\} -- команда загрузки дополнительных пакетов, с её помощью можно добавить поддержку кириллицы в документ или подключить математический модуль, чтобы легко записывать формулы. 
    \item author\{...\} -- команда, с помощью которой задается имя автора документа.
    \item title\{...\} -- команда, с помощью которой указывается название документа.
    \item date\{...\} -- каманда, с помощью которой можно добавить дату в документ.
    \item begin\{document\} -- команда, с помощью которой задается начало документа, записанные после begin команды будут отображаться в документе.
    \item end\{document\} -- команда, с помощью которой задается конец документа.
\end{enumerate}


\end{document} % конец документа

